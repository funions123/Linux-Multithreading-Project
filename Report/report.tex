\documentclass[12pt]{article}

\usepackage{cite}
\usepackage{hyperref}

\title{Linux Multithreading Demo}
\author{Thomas Powell}
\date{\today}

\begin{document}
\maketitle
\pagebreak
\section{Introduction}

\paragraph{}
tpowel86@students.kennesaw.edu

CS3502 Section 02

\paragraph{}
This project is a simple banking simulation that uses multithreading to handle transactions concurrently.
Accounts are protected by mutexes, and transactions are distributed to worker threads.
A subprocess is used to log the results of transactions, receiving the log output from the worker threads via a pipe.

\section{Implementation}

\paragraph{}
For this project, I created a basic banking simulation that manages an arbitrary number of bank accounts, and has a thread manager to handle transactions.
Both the accounts and transactions are stored in JSON files, which are parsed and then executed by the program.
The main thread is responsible for distributing transactions to worker threads. 
The number of worker threads is defined arbitrarily, and transactions are distributed round-robin.
Worker threads execute the queued transactions on accounts, and each account stores a mutex to lock access to it while any thread is executing a transaction.
Unless blocked by a mutex, the worker threads will execute all transactions as fast as possible.

\paragraph{}
For demonstrating separate process and pipes, I created a subprocess that handles logging the results of transactions.
This subprocess is started by the main process, and the main process writes the results to a pipe.
The subprocess reads the pipe, and prints the results to a text file and the console.
While most of the program will work on both windows and linux, the pipe only works on Linux.

\section{Environment Setup}

\paragraph{}
For this project, I used VirtualBox and the latest stable version of Ubuntu.
This made things relatively simple, as I could install VSCode, Github Desktop, and all required libraries and utilities through relatively conventional means.
In VSCode's case, it was even on the app store.
After I installed all of my extensions into VSCode, I could compile the program using GCC/G++ from VSCode.
VSCode even has extensions for editing \LaTeX\ files, which I found very helpful.
This essentially meant my entire workflow was inside VSCode, and not particularly different from what I am used to on Windows.
The \LaTeX\  extensions even auto-compiled valid files to pdf every time I saved.

\paragraph{}
Another very useful tool I used was the nlohmann/json library, which I used to parse and handle JSON data.
This library is very easy to use, and I have a lot of experience with it due to my other projects.
You can install it by drag and dropping the json.hpp file into the workspace, without having to use conan or another lib manager. 
You don't even need to configure cmake for an external library, since cmake will include it the normal way.

\section{Challenges}

\paragraph{}
One of the first challenges that appeared as I was setting up the project was my total unfamilarity with \LaTeX. 
Luckily, there are many great resources online that helped me learn how to use it.
NYU's sample document\cite{1} and "A Gentle Introduction to \TeX\ A Manual for Self-study.”\cite{2} were the main resources I read from.
VSCode turned out to have many useful extensions for \LaTeX, so I was able to write and compile the document inside VSCode.

\paragraph{}
Another major challenge I ran into was the nature of mutexes.
My initial implementation didn't work, since mutexes are not copyable.
This was a problem because I was trying to pass the mutex to the worker threads, and the accounts were being copied.
I was able to fix this by making the accounts shared pointers, and storing them in a global list.
This meant that I never copied the accounts, since the global list was accessed using extern.

\section{Conclusion}

\paragraph{}
Once I got everything working, the program worked as expected.
I was able to deposit and withdraw money from the accounts, and the program handled the multithreaded transactions correctly.
The logging subprocess worked as expected, and the results were printed to the console and a text file.

\paragraph{}
This project was a great learning experience for me, and I learned a lot about everything involved.
Despite being experienced in C++, I had never used threads or mutex before.
My experience with Linux and VMs was also very limited before this project.

\begin{thebibliography}{1}

\bibitem{1} 
    ``Research Guides: Getting Started with LaTeX: A Sample Document,''
    Nyu.edu 2024.
    Accessed: Feb. 23, 2025.
    Available: \url{https://guides.nyu.edu/LaTeX/sample-document}

\bibitem{2}
    M. Doob, 
    “A Gentle Introduction to \TeX A Manual for Self-study.”,
    Accessed: Feb. 23, 2025. 
    Available: \url{https://texdoc.org/serve/gentle.pdf/0}



\end{thebibliography}

\end{document}